%\documentclass[a4paper,twoside,10pt,openright]{report}%Dokumentenklasse 
\usepackage{graphicx} %Compiler

\usepackage{dsfont}

%%%% Schrift und Kodierung %%%%
	\usepackage[T1]{fontenc} %Zeichensatzkodierung von 7bit auf 8bit 
	\usepackage[utf8]{inputenc} %Zeichensatzkodierung Unicode bzw. UTF8
	\usepackage{lmodern} %Vektorschrift
	%\RequirePackage[english,spanish,es-nolayout]{babel}
	\usepackage[english]{babel}
	\usepackage{textcomp}
	\usepackage{lmodern}
	\usepackage{url}
	\usepackage[bookmarksnumbered=true]{hyperref}
    \graphicspath{{figures/}}	
	
%%% COLORS %%%
\newcommand{\hl}[1]{
   \textcolor{MidnightBlue!90!black}{#1} 
}

\usepackage{pifont}% http://ctan.org/pkg/pifont
\newcommand{\cmark}{\ding{51}}%
\newcommand{\xmark}{\ding{55}}%

\usepackage{stackrel}

\usepackage{feynmp}

\usepackage{amsthm}
\newtheorem{definition}{Definition}
\newtheorem{proposition}{Proposition}
	
%%%% Fancy Header %%%
\usepackage{fancyhdr}
\usepackage[dvipsnames]{xcolor}
\usepackage{tikz} 
\usepackage{pgfplots}
%\usepackage{pgf-pie}

\usetikzlibrary{arrows}
\usetikzlibrary{shapes.geometric, arrows}

\definecolor{mycolor}{rgb}{0.45,0.45,0.45}% dark grey

\newcommand{\mcD}{\mathcal{D}}
\newcommand{\mcL}{\mathcal{L}}
\newcommand{\dth}{\Delta_{\text{th/sys}}}
\newcommand{\dst}{\Delta_{\text{stat}}}
\newcommand{\dsy}{\Delta_{\text{syst}}}
\newcommand{\sthsy}{\sigma_{\text{th/sys}}}
\newcommand{\sth}{\sigma_{\text{th}}}
\newcommand{\sst}{\sigma_{\text{stat}}}
\newcommand{\ssy}{\sigma_{\text{syst}}}

\newcommand{\Langle}{\big\langle}
\newcommand{\Rangle}{\big\rangle} 
 
\fancyhf{}
\fancyhead[LE]{\sffamily\color{mycolor}\nouppercase{\leftmark}} % left even, right odd
\fancyhead[RO]{\sffamily\color{mycolor}\nouppercase{\rightmark}} % left even, right odd
\fancyfoot[CE,CO]{\sffamily\color{mycolor}\nouppercase{\thepage}} % center even, center odd
\renewcommand{\headrule}{{\color{mycolor}%
\hrule width\headwidth height\headrulewidth \vskip-\headrulewidth}}
\renewcommand{\headrulewidth}{0.5pt}

\fancypagestyle{plain}{%
    \fancyhf{}%
    \fancyfoot[CE,CO]{ { \sffamily\color{mycolor}{\thepage} } }
	\renewcommand{\headrulewidth}{0.0pt}
}

%\fancypagestyle{fancy}{%
%   \fancyhf{}
%	\fancyhead[LE,RO]{\sffamily\color{mycolor}\nouppercase{\leftmark}} % left even, right odd
%	\fancyfoot[CE,CO]{\sffamily\color{mycolor}\nouppercase{\thepage}} % center even, center odd
%	\renewcommand{\headrule}{{\color{mycolor}%
%	\hrule width\headwidth height\headrulewidth \vskip-\headrulewidth}}
%	\renewcommand{\headrulewidth}{0.5pt}
%}


%%% Customize titles %%%%
\usepackage[ ]{titlesec}  %
\usepackage{etoolbox}
\makeatletter
\patchcmd{\ttlh@hang}{\parindent\z@}{\parindent\z@\leavevmode}{}{}
\patchcmd{\ttlh@hang}{\noindent}{}{}{}
\makeatother
%\titleformat{\chapter}[display]
%  { \normalsize \huge  \color{black}}%
%  {\flushright \normalsize \color{mycolor} \MakeUppercase %
%  {\sffamily \chaptertitlename } \hspace{1 ex}%
%  { \fontsize{60}{60}\selectfont \color{mycolor} \sffamily  \thechapter }}%
%  {10 pt}%
%  {\sffamily \huge \color{mycolor}\bfseries}
%\newcommand\mychapformat[1]{\parbox[t]{\dimexpr\textwidth-3em\relax}{\raggedleft#1}}
\titleformat{\chapter}[hang]{\Huge\bfseries\color{mycolor}\sffamily}% the number
	{\thechapter\hspace{20pt}\textcolor{mycolor}{|}\hspace{20pt}}%
	{0pt}{\Huge\bfseries\color{mycolor}\sffamily}% the title
\titleformat*{\section}{\sffamily\LARGE\color{mycolor}}
\titleformat*{\subsection}{\sffamily\Large\color{mycolor}}
\titleformat*{\subsubsection}{\sffamily\large\color{mycolor}}
\titleformat*{\paragraph}{\sffamily\large\bfseries\color{mycolor}}



	
%%%% Mathepakete %%%%
	\usepackage{array}
	\usepackage{calc}
	\usepackage{amsmath}
	\usepackage[intlimits]{empheq}
	\usepackage{amssymb,mathrsfs}
	\usepackage{theorem}
	\usepackage{slashed}
	\usepackage{feynmp-auto}

%%%% Sonstiges %%%%
	%\usepackage{subcaption}
	\expandafter\def\csname ver@subfig.sty\endcsname{}
	\usepackage{subfig} %Ermöglicht subfloats, also mehrere Tabellen/Bilder in einer Umgebung
	\usepackage{float} %Setzt mit [H] Figuren genau dort hin, wo sie im Text auftauchen
	\usepackage{booktabs} %Andere Tabellen
	\usepackage{gensymb}
	\usepackage{extarrows}%lange Pfeile
%	\usepackage{pst-pdf}	
	\usepackage{wasysym} %Symbolpaket
	\usepackage{multirow} %Ein Wert für mehrere Zeilen oder Spalten von Tabellen. Verwendung \multirow{#AnzahlZeilen}{*}{Name} bzw. analog \multicolumn{}{}{}
	\usepackage{rotating} %ermöglicht Schiefe Schrift
	%\usepackage{ziffer} %Deutsche Zahlen (Komma als Dezimalstelle im Mathemodus!)
	\usepackage{nicefrac} %Im Text schöne Brüche
	\usepackage[inner=3cm, outer=2.4cm, top=3cm]{geometry} %Passt Seitenränder an (left, right, top, bottom, width, height, textwidth, textheight)
	\usepackage{scrhack} %Verbessert angeblich LaTeX-Pakete
	\usepackage{numprint} %ROOT-Zahlen in deutsches zahlenformat übertragen. Syntax: \numprint[kg]{1.234e56} wird zu 1,234 * 10^56 kg
	\usepackage{cite}
	\usepackage{placeins}
	\usepackage{changepage}
	%\hyphenation{con-fine-ment}
	
%%%% spezielle Formatierungen %%%%
	\setlength{\emergencystretch}{25pt} %verhindert das Herausragen von Wörtern übers Zeilenende
	\setlength{\parindent}{0pt} %Kein Einschub bei neuem Absatz
	\setlength{\parskip}{2pt plus 1pt} %Erhöht Abstand zwischen Absätzen (um 1pt flexibel bei Seitenumbrüchen)
	
	%%%% Dokument-Variablen %%%%
\date{\today}

%%%% Eigene Befehle %%%%
\DeclareGraphicsRule{*}{mps}{*}{}
	\newcommand{\mE}[1]{\,\mathrm{#1}} %Einheiten im Mathemodus
	\renewcommand{\sl}[1]{\slashed{#1}} %Feynman-Slash
	\newcommand{\dummyImage}[2]{  %Erzeugt eine Umgebung wie includegraphics
		\frame{\mbox{\rule{0pt}{#2}	Bild fehlt  noch \rule{#1}{0pt}}}
	}
	\renewcommand{\i}{\mathrm{i}} %Imaginäre Einheit
%	\renewcommand{\vec}[1]{\textbf{#1}} %Fette Vektoren
	\newcommand{\bc}{\begin{center}}
	\newcommand{\ec}{\end{center}}

\newcommand{\matM}{\mathcal{M}}
\newcommand{\matH}{\mathcal{H}}
\newcommand{\matS}{\mathcal{S}}
\newcommand{\matU}{\mathcal{U}}
\newcommand{\matUL}{\mathcal{U}_L}
\newcommand{\matUR}{\mathcal{U}_R}

%%% center all the figures and tables
\makeatletter
\g@addto@macro\@floatboxreset\centering
\makeatother

%% maximal number of floating environments on each page 
\setlength{\floatsep}{0pt}
\setcounter{topnumber}{1}
\setcounter{bottomnumber}{1}
\setcounter{totalnumber}{1}
\renewcommand{\topfraction}{1.0}
\renewcommand{\bottomfraction}{1.0}
\renewcommand{\textfraction}{0.0}
\renewcommand{\thefootnote}{\fnsymbol{footnote}}

\def\tablename{Table}
\def\figurename{Figure}

\newcommand{\newparagraph}{\par\bigskip\noindent}
\newcommand{\toolfont}[1]{\texttt{#1}}
\newcommand{\ord}{\ensuremath{\mathcal{O}}}
\newcommand{\ope}[1]{\ensuremath{\mathcal{O}_{#1}}}
\newcommand{\largex}{\ensuremath{\large \boldsymbol{\times}}}
\newcommand{\brlargex}{\ensuremath{\large (\boldsymbol{\times})}}
\newcommand{\mheavy}{\ensuremath{M}}

\newcommand{\lag}{\ensuremath{\mathcal{L}}}
\newcommand{\mat}{\ensuremath{\mathcal{M}}}
\newcommand{\delx}{\ensuremath{\Delta}}
\newcommand{\data}{\ensuremath{\mathcal{D}}}
\newcommand{\jump}{\vspace{0.3cm}}
\newcommand{\bsg}{\ensuremath{\mathcal{B}(b \to s \gamma)}}
\newcommand{\rself}{\ensuremath{\hat{\Sigma}}}
\newcommand{\retildehat}{\ensuremath{\mbox{Re}\hat{\Sigma}}}
\newcommand{\retilde}{\ensuremath{\mbox{Re}\,\Sigma}}
\newcommand{\dweaksing}{\ensuremath{\delta_{\text{weak}}^{\text{sing}}}}
\newcommand{\dweak}{\ensuremath{\delta_{\text{weak}}}}
\newcommand{\newtext}[1]{\textcolor{red}{#1}}
\newcommand{\suit}{\textcolor{blue}{$\spadesuit$}}
\newcommand{\neutn}{\ensuremath{\tilde{\chi}^0_n}}
\newcommand{\gluino}{\ensuremath{\tilde{g}}}
\newcommand{\squark}{\ensuremath{\tilde{q}}}
\newcommand{\met}{\ensuremath{\slashed{E}_T}}
\newcommand{\nqsq}{\ensuremath{\tilde{\chi}\,\tilde{q}\,q}}
\newcommand{\msbar}{\ensuremath{\overline{MS}}}
\newcommand{\sw}{\ensuremath{s_w}}
\newcommand{\swd}{\ensuremath{s^2_w}}
\newcommand{\cw}{\ensuremath{c_w}}
\newcommand{\cwd}{\ensuremath{c^2_w}}
\newcommand{\myrbox}[1]{\parbox{4.0cm}{#1}}

\usepackage{xspace}
\newcommand{\brinv}{\ensuremath{BR_{\text{inv}}}\xspace}
\newcommand{\vegas}{\textsc{Vegas}\xspace}
\newcommand{\madgraph}{\textsc{Madgraph}\xspace}
\newcommand{\geant}{\textsc{Geant4}\xspace}
\newcommand{\pythia}{\textsc{Pythia}8\xspace}
\newcommand{\fastjet}{\textsc{FastJet}\xspace}
\newcommand{\delphes}{\textsc{Delphes}\xspace}
\newcommand{\sherpa}{\textsc{Sherpa}\xspace}
\newcommand{\sklearn}{\textsc{scikit-learn}\xspace}
\newcommand{\keras}{\textsc{Keras}\xspace}
\newcommand{\tensorflow}{\textsc{TensorFlow}\xspace}
\newcommand{\pytorch}{\textsc{PyTorch}\xspace}
\newcommand{\theano}{\textsc{Theano}\xspace}
\newcommand{\adam}{\textsc{Adam}\xspace}

\newcommand{\psib}{\overline{\psi}}
\newcommand{\bpm}{\begin{pmatrix}}
\newcommand{\epm}{\end{pmatrix}}

\newcommand{\p}{\partial}
\newcommand{\br}{\text{BR}}
\newcommand{\qqquad}{\qquad \qquad}
\newcommand{\qqqquad}{\qquad \qquad \qquad}

\newcommand{\matx}{|\mathcal{M}|^2}
\newcommand{\really}{\stackrel{!}{=}}
\newcommand{\SFitter}{\textsc{SFitter} }

% units of measure
\newcommand{\mev}{{\ensuremath\rm MeV}}
\newcommand{\gev}{{\ensuremath\rm GeV}}
\newcommand{\tev}{{\ensuremath\rm TeV}}
\newcommand{\fb}{{\ensuremath\rm fb}}
\newcommand{\ab}{{\ensuremath\rm ab}}
\newcommand{\pb}{{\ensuremath\rm pb}}
\newcommand{\sign}{{\ensuremath\rm sign}}
\newcommand{\iab}{\text{ab}^{-1}}
\newcommand{\ifb}{{\ensuremath\rm fb^{-1}}}
\newcommand{\ipb}{{\ensuremath\rm pb^{-1}}}

% really great macro by Chris Lester
\def\slashchar#1{\setbox0=\hbox{$#1$}           % set a box for #1
   \dimen0=\wd0                                 % and get its size
   \setbox1=\hbox{/} \dimen1=\wd1               % get size of /
   \ifdim\dimen0>\dimen1                        % #1 is bigger
      \rlap{\hbox to \dimen0{\hfil/\hfil}}      % so center / in box
      #1                                        % and print #1
   \else                                        % / is bigger
      \rlap{\hbox to \dimen1{\hfil$#1$\hfil}}   % so center #1
      /                                         % and print /
   \fi}
\newcommand{\dslash}{\slashchar{\partial}}
\newcommand{\Dslash}{\slashchar{D}}

\def\eg{{e.g.}\ }
\def\ie{{i.e.}\ }
%\def\etal{{\sl et al} \,}
%\DeclareMathOperator{\tr}{Tr}
\newcommand{\pbp}{\ensuremath{H^\dagger\,H}}
\DeclareMathOperator{\tr}{Tr}
\newcommand{\Dfb}{\mbox{$\raisebox{2mm}{\boldmath ${}^\leftrightarrow$}\hspace{-4mm} D$}}
\newcommand{\Dfba}{\mbox{$\raisebox{2mm}{\boldmath ${}^\leftrightarrow$}\hspace{-4mm} D^a$}}
\newcommand{\overbar}[1]{\mkern 1.5mu\overline{\mkern-1.5mu#1\mkern-1.5mu}\mkern 1.5mu}
\let\vec\mathbf % vectors in bold
\renewcommand{\d}{\text{d}}


%\pagestyle{fancy}
%\graphicspath{{../figures/}}	
%\begin{document}

\chapter{Introduction}\label{chap:introduction}
\enlargethispage{2ex}
\vspace*{-2pt}
The motivation behind the prediction of a fundamental scalar particle in the Standard Model (SM), 
the Higgs boson, was to grant
a mechanism for the generation of the masses of the electroweak gauge bosons via 
electroweak symmetry breaking (EWSB)~\cite{Higgs:1964pj,Higgs:1964ia,Englert:1964et}. 
The discovery of a Higgs boson at the Large Hadron Collider (LHC)~\cite{Aad:2012tfa,Chatrchyan:2012xdj} 
%and precision analyses of its properties 
strongly hints at EWSB indeed being the mechanism behind the mass generation of the SM particles.
One of the pivotal tasks of the LHC and future colliders is to probe both the local and global structure of the Higgs 
potential,
%While after EWSB the local properties of the Higgs potential in the vicinity of the electroweak vacuum 
which is reflected in the couplings of the Higgs boson to other SM particles and in its
self-coupling, respectively. 
%With the end of LHC Run~II and its experimental legacy coming in, 
In this thesis, we present a global view on Higgs couplings at the LHC
to extend our understanding of the EWSB sector and
to set universal constraints on new physics that might be hiding in it.
\\ \medskip \vspace*{-2pt}


The couplings of the Higgs boson to other SM particles manifest the 
local properties of its potential in the 
vicinity of the electroweak vacuum after EWSB. 
LHC measurements of the various predicted Higgs production and decay channels are 
crucial to explore and constrain these couplings. 
So far, the (preliminary) results of LHC Run~II are compatible with the 
couplings predicted for the SM Higgs boson~\cite{ATLAS-CONF-2019-005,Sirunyan:2018koj}.
Its four dominant production modes at the LHC 
%gluon fusion, %~\cite{Aad:2012tfa,Chatrchyan:2012xdj}, 
%weak boson fusion, production in association with a vector boson and 
%in association with a pair of top quarks %~\cite{Aaboud:2018urx,}
have been observed with no significant deviation from the SM expectations.
Moreover, LHC Run~II has established the Higgs decays into the 
kinematically accessible third generation fermions in addition to the decays
into pairs of the electroweak gauge bosons.
%a pair of bottom-quarks, $\tau$~leptons, %~\cite{Sirunyan:2017khh}, 
%photons as well as $Z$~bosons and $W$~bosons have been observed
Tight constraints have been set on the Higgs branching ratio to a pair of muons, to $Z \gamma$
or to \textit{invisible}, \ie to undetectable, particles. 


While single-Higgs production measurements probe the local structure of the Higgs potential
and provide only indirect constraints on the realization of EWSB, 
the examination of the global structure of the potential requires (at least) di-Higgs production. 
%As a function of the symmetry breaking potential’s parameters, 
%this production mode is
%highly sensitive to the realization of electroweak symmetry breaking.
%
This process is sensitive to the trilinear Higgs self-coupling, which the
LHC will only constrain to 
multiple times its SM value, even after its high-luminosity run~\cite{Kim:2018uty}. 
For measurements in the percent range, future colliders are \mbox{required~\cite{DiVita:2017vrr,Goncalves:2018yva}}. 
%
In any collider experiment, precise measurements of Higgs couplings to other SM particles are
a crucial ingredient for the extraction of Higgs self-coupling from multi-Higgs 
production~\cite{DiVita:2017vrr,Kim:2018uty,Biekotter:2018jzu},
which emphasizes the relevance 
of the local properties of the Higgs potential 
on a global scale. 


The exploration of the structure of the Higgs potential is not only indispensable
to gain a deeper understanding of EWSB on a fundamental level, 
it also provides important constraints for physics beyond the SM (BSM).
The motivations for extension of the SM Higgs sector and their impact on
Higgs couplings are versatile: 
%
The Higgs boson might be the mediator to a dark sector~\cite{Schabinger:2005ei,Barbieri:2005ri,Patt:2006fw,
Bertolami:2007wb,Gonderinger:2009jp,Andreas:2010dz,Tytgat:2010bt,Englert:2011yb,
Pospelov:2011yp,He:2011de,Fox:2011pm,Low:2011kp,Englert:2011aa,Djouadi:2011aa,
Batell:2011pz,Baek:2014jga,Butter:2015fqa,Banerjee:2016nzb,Barman:2017swy}
resulting in, for instance, an increased Higgs branching ratio to invisible particles and global 
rescalings of its couplings to other SM particles. 
%
A modified Higgs potential or an extended Higgs sector has direct implications for 
vacuum stability~\cite{Sher:1988mj,Casas:1996aq,Degrassi:2012ry,Bednyakov:2015sca} 
and 
electroweak baryogenesis~\cite{Grojean:2004xa,Delaunay:2007wb,Noble:2007kk,Huang:2015tdv,
Kobakhidze:2015xlz,Assamagan:2016azc,Chen:2017qcz,
Gan:2017mcv,Cao:2017oez,Jain:2017sqm,deVries:2017ncy,Reichert:2017puo,Carena:2018vpt}, 
which requires a strong first-order phase transition to 
ensure a deviation from thermal equilibrium~\cite{Sakharov:1967dj}.
If in fact electroweak baryogenesis is the mechanism responsible for the baryon-antibaryon asymmetry 
in the universe, the remnants of the electroweak phase transition may have 
observable consequences, \eg,  in the form of gravitational waves~\cite{Kosowsky:1992rz,Dolgov:2002ra,
Delaunay:2007wb,Chala:2019rfk}.
%
%to inflation~\cite{Bezrukov:2007ep}.
%
%In the SM, the cross section of di-Higgs production is rendered accidentally small 
%by the negative interference between its two dominant diagrams.  
%This process is hence very sensitive to BSM physics. 
\\ \medskip \vspace*{-2pt}


%Driven by the question what we already know about the  
%EWSB sector and the new physics that might be hiding in it, 
%we set limits on Higgs couplings in global fit.
%Furthermore, we will point out approaches to improve these limits by making more efficient 
%use of the available data as well as by probing the Higgs sector at a potential future upgrade
%of the LHC to an energy of 27~TeV. \\ \medskip

Driven by the question what current data reveal about the  
EWSB sector and new physics that might be hiding in it, in this thesis 
we aim at increasing the precision of Higgs-coupling measurements 
and combining them in a comprehensive framework. 
This requires us to rethink the way we perform,  
interpret and combine experimental analyses in a way that fully exploits the available data. 
To tackle this challenge, we take a multi-prong approach:
First, we focus on the improvement of an individual Higgs-production and decay
channel by applying modern analysis techniques. Second, we perform global 
analyses of the Higgs-gauge sector for the LHC and a potential future upgrade 
of the LHC in a model-independent framework.
\\ \medskip \vspace*{-2pt}


Data driven analysis techniques are applied to the experimental analyses 
of individual search channels more and more frequently. 
They replace simple cut-and-count strategies and vetoes by more advanced multivariate  
analyses and machine learning to profit from to the full
information provided by the data. 
A prime test bed for the application of these new approaches is given by jets,
not only because the LHC generates 
ample of them, but also because their substructure relies on 
relatively simple physical principles.
In Chapter~\ref{cha:InvHiggs}, we apply a multivariate analysis 
to the tagging jets in weak-boson-fusion Higgs production with an invisible 
decay of the Higgs boson. 
%This channel is the most sensitive to constrain 
%an invisible Higgs decays at the LHC, a signature which is strongly motivated 
%from a BSM perspective, as mentioned above. 
%
Based on the observation that the tagging jets in the weak-boson-fusion Higgs signal are more likely
to be quark-initiated, 
we examine the potential of variables targeting quark/gluon discrimination 
to suppress the gluon-dominated QCD backgrounds.
%
%In the SM, invisible Higgs decays only arises from a Higgs decay to a pair of $Z$~bosons and their
%subsequent decay to neutrinos, $p p \rightarrow Z Z^* \rightarrow 4 \nu$, 
%yielding a SM branching ratio for this decay mode of only $1$\textperthousand.
%This is clearly beyond the experimental reach which currently constrains an invisible Higgs 
%branching ratio to below $19$\%~\cite{Sirunyan:2018owy}.
%However, as mentioned above, invisible Higgs decays are strongly 
%motivated from a BSM perspective.
% for instance in the context of 
%dark matter or electroweak baryogenesis. 
%Analyses of invisible Higgs decays are experimentally challenging and current 
%limits on 
%The observation of the SM Higgs branching ratio into invisible particles is beyond
%the reach of the LHC or future colliders. Even with a 100~TeV collider, we 
%could only set an upper limit a bit below the one percent. 
%
\\ \medskip \vspace*{-2pt}

%In principle, we can look for new physics both in targeted searches for 
%the exotic signatures predicted by (UV-complete) theories of physics beyond the 
%Standard Model as well as through their modifications of the interactions 
%anticipated in the SM.  
%As a result of experimental null results in \textit{smoking gun} signatures, however, 
%analyses are now increasingly focussing on precision measurements of SM quantities. 
%

An economic use of the available Higgs data calls for a combination of
measurements from different experiments, sectors 
and scales. Such comprehensive study can aid in making small effects of new physics 
visible on a global scale and demands for a universal parametrization of those. 
Historically, deviations from the SM Higgs couplings were described 
by coupling modifiers in the $\Delta$-framework~\cite{Lafaye:2009vr} (or the closely related 
$\kappa$-framework introduced in Ref.~\cite{LHCHiggsCrossSectionWorkingGroup:2012nn}). 
A phenomenologically more powerful framework to 
probe the data %from the LHC and elsewhere  
for hints of possible BSM physics 
in an almost model-independent way 
is given by SM effective field theory (SMEFT)~\cite{Weinberg:1978kz,Leung:1984ni,Buchmuller:1985jz,
GonzalezGarcia:1999fq,Grzadkowski:2010es,Passarino:2012cb}, 
introduced in Chapter~\ref{cha:SMEFT}.
It directly links the Higgs and gauge sectors and allows for the modelling of 
modified Lorentz structures. 
We confront the SMEFT framework with data using the fitting 
tool \textsc{SFitter}~\cite{Lafaye:2007vs}.
As discussed in Chapter~\ref{cha:global}, \textsc{SFitter} allows for an exhaustive treatment
of statistical, systematic and theoretical uncertainties as well as their correlations.
%\\ \medskip

Motivated by the experimental advances of LHC Run~II, we perform a global fit of the Higgs-gauge sector 
based on Higgs and di-boson measurements as well as electroweak precision data in Chapter~\ref{chap:fit13TeV}. 
We include momentum-related kinematic distributions 
%for Higgs production in association with a vector 
%boson and for di-boson production. 
and examine the impact of the different LHC Run~II measurements on the reach of our 
global analysis in detail. 
On the theory side, we broaden our view on the Higgs sector by expanding 
the set of considered dimension-six operators from~10 to~18 with respect to 
previous \textsc{SFitter} analyses~\cite{Corbett:2015ksa,Butter:2016cvz}.
This extension of our operator set will bring 
us a significant step closer to a global SMEFT fit at dimension six.
%The  considered are constrained by electroweak precision data.
We discuss how the additional fermionic Higgs-gauge operators have a relevant 
impact on a global fit of the Higgs-gauge sector
despite the strong constraints from electroweak precision data.

An upgrade of the LHC to an energy of 27~TeV is among the realistic proposals 
for future colliders following the high-luminosity LHC era. 
The capability of such a 27~TeV hadron collider to produce a statistically 
relevant number of di-Higgs events prompts us to perform a global fit 
of the Higgs-gauge sector including a modified Higgs potential.
In Chapter~\ref{chap:fit27TeV}, we assess the sensitivity of a high-energy upgrade 
of the LHC to the Wilson coefficients of dimension-six operators in the SMEFT framework. 
We thoroughly examine the correlations of operators influencing 
the extraction of the trilinear Higgs self-coupling and thereby probe 
the relevance of precise constraints on the local properties of the Higgs 
potential for the study of its global structure.
\\ \medskip \vspace*{-2pt}





In Chapter~\ref{cha:conclusion}, we will summarize our results and give an 
outlook to further improvements and extensions of the concepts discussed 
in this thesis. 
%
Each of the lines of research mentioned above will aid in constructing a global view of Higgs 
couplings at the LHC as well as its proposed future upgrade 
and will bring us one step closer to probing if EWSB
is indeed described by the simple structure of the SM Higgs potential.
The derived limits on Higgs couplings in the SMEFT framework 
can be mapped onto constraints for UV-complete BSM models~\cite{deBlas:2017xtg,Ellis:2018gqa}.
Furthermore, they provide a key ingredient for future tests of the global structure of the Higgs potential. 
In summary, the thorough investigation of Higgs couplings at the LHC is crucial 
to gain a deeper understanding of the structure of the Higgs sector and EWSB on a fundamental level.
\enlargethispage{2ex}


%\end{document}
