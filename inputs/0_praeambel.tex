\documentclass[a4paper,twoside,10pt,openright]{report}%Dokumentenklasse 
\usepackage{graphicx} %Compiler

\usepackage{dsfont}

%%%% Schrift und Kodierung %%%%
	\usepackage[T1]{fontenc} %Zeichensatzkodierung von 7bit auf 8bit 
	\usepackage[utf8]{inputenc} %Zeichensatzkodierung Unicode bzw. UTF8
	\usepackage{lmodern} %Vektorschrift
	%\RequirePackage[english,spanish,es-nolayout]{babel}
	\usepackage[english]{babel}
	\usepackage{textcomp}
	\usepackage{lmodern}
	\usepackage{url}
	\usepackage[bookmarksnumbered=true]{hyperref}
    \graphicspath{{figures/}}	
	
%%% COLORS %%%
\newcommand{\hl}[1]{
   \textcolor{MidnightBlue!90!black}{#1} 
}

\usepackage{pifont}% http://ctan.org/pkg/pifont
\newcommand{\cmark}{\ding{51}}%
\newcommand{\xmark}{\ding{55}}%

\usepackage{stackrel}

\usepackage{feynmp}

\usepackage{amsthm}
\newtheorem{definition}{Definition}
\newtheorem{proposition}{Proposition}
	
%%%% Fancy Header %%%
\usepackage{fancyhdr}
\usepackage[dvipsnames]{xcolor}
\usepackage{tikz} 
\usepackage{pgfplots}
%\usepackage{pgf-pie}

\usetikzlibrary{arrows}
\usetikzlibrary{shapes.geometric, arrows}

\definecolor{mycolor}{rgb}{0.45,0.45,0.45}% dark grey

\newcommand{\mcD}{\mathcal{D}}
\newcommand{\mcL}{\mathcal{L}}
\newcommand{\dth}{\Delta_{\text{th/sys}}}
\newcommand{\dst}{\Delta_{\text{stat}}}
\newcommand{\dsy}{\Delta_{\text{syst}}}
\newcommand{\sthsy}{\sigma_{\text{th/sys}}}
\newcommand{\sth}{\sigma_{\text{th}}}
\newcommand{\sst}{\sigma_{\text{stat}}}
\newcommand{\ssy}{\sigma_{\text{syst}}}

\newcommand{\Langle}{\big\langle}
\newcommand{\Rangle}{\big\rangle} 
 
\fancyhf{}
\fancyhead[LE]{\sffamily\color{mycolor}\nouppercase{\leftmark}} % left even, right odd
\fancyhead[RO]{\sffamily\color{mycolor}\nouppercase{\rightmark}} % left even, right odd
\fancyfoot[CE,CO]{\sffamily\color{mycolor}\nouppercase{\thepage}} % center even, center odd
\renewcommand{\headrule}{{\color{mycolor}%
\hrule width\headwidth height\headrulewidth \vskip-\headrulewidth}}
\renewcommand{\headrulewidth}{0.5pt}

\fancypagestyle{plain}{%
    \fancyhf{}%
    \fancyfoot[CE,CO]{ { \sffamily\color{mycolor}{\thepage} } }
	\renewcommand{\headrulewidth}{0.0pt}
}

%\fancypagestyle{fancy}{%
%   \fancyhf{}
%	\fancyhead[LE,RO]{\sffamily\color{mycolor}\nouppercase{\leftmark}} % left even, right odd
%	\fancyfoot[CE,CO]{\sffamily\color{mycolor}\nouppercase{\thepage}} % center even, center odd
%	\renewcommand{\headrule}{{\color{mycolor}%
%	\hrule width\headwidth height\headrulewidth \vskip-\headrulewidth}}
%	\renewcommand{\headrulewidth}{0.5pt}
%}


%%% Customize titles %%%%
\usepackage[ ]{titlesec}  %
\usepackage{etoolbox}
\makeatletter
\patchcmd{\ttlh@hang}{\parindent\z@}{\parindent\z@\leavevmode}{}{}
\patchcmd{\ttlh@hang}{\noindent}{}{}{}
\makeatother
%\titleformat{\chapter}[display]
%  { \normalsize \huge  \color{black}}%
%  {\flushright \normalsize \color{mycolor} \MakeUppercase %
%  {\sffamily \chaptertitlename } \hspace{1 ex}%
%  { \fontsize{60}{60}\selectfont \color{mycolor} \sffamily  \thechapter }}%
%  {10 pt}%
%  {\sffamily \huge \color{mycolor}\bfseries}
%\newcommand\mychapformat[1]{\parbox[t]{\dimexpr\textwidth-3em\relax}{\raggedleft#1}}
\titleformat{\chapter}[hang]{\Huge\bfseries\color{mycolor}\sffamily}% the number
	{\thechapter\hspace{20pt}\textcolor{mycolor}{|}\hspace{20pt}}%
	{0pt}{\Huge\bfseries\color{mycolor}\sffamily}% the title
\titleformat*{\section}{\sffamily\LARGE\color{mycolor}}
\titleformat*{\subsection}{\sffamily\Large\color{mycolor}}
\titleformat*{\subsubsection}{\sffamily\large\color{mycolor}}
\titleformat*{\paragraph}{\sffamily\large\bfseries\color{mycolor}}



	
%%%% Mathepakete %%%%
	\usepackage{array}
	\usepackage{calc}
	\usepackage{amsmath}
	\usepackage[intlimits]{empheq}
	\usepackage{amssymb,mathrsfs}
	\usepackage{theorem}
	\usepackage{slashed}
	\usepackage{feynmp-auto}

%%%% Sonstiges %%%%
	%\usepackage{subcaption}
	\expandafter\def\csname ver@subfig.sty\endcsname{}
	\usepackage{subfig} %Ermöglicht subfloats, also mehrere Tabellen/Bilder in einer Umgebung
	\usepackage{float} %Setzt mit [H] Figuren genau dort hin, wo sie im Text auftauchen
	\usepackage{booktabs} %Andere Tabellen
	\usepackage{gensymb}
	\usepackage{extarrows}%lange Pfeile
%	\usepackage{pst-pdf}	
	\usepackage{wasysym} %Symbolpaket
	\usepackage{multirow} %Ein Wert für mehrere Zeilen oder Spalten von Tabellen. Verwendung \multirow{#AnzahlZeilen}{*}{Name} bzw. analog \multicolumn{}{}{}
	\usepackage{rotating} %ermöglicht Schiefe Schrift
	%\usepackage{ziffer} %Deutsche Zahlen (Komma als Dezimalstelle im Mathemodus!)
	\usepackage{nicefrac} %Im Text schöne Brüche
	\usepackage[inner=3cm, outer=2.4cm, top=3cm]{geometry} %Passt Seitenränder an (left, right, top, bottom, width, height, textwidth, textheight)
	\usepackage{scrhack} %Verbessert angeblich LaTeX-Pakete
	\usepackage{numprint} %ROOT-Zahlen in deutsches zahlenformat übertragen. Syntax: \numprint[kg]{1.234e56} wird zu 1,234 * 10^56 kg
	\usepackage{cite}
	\usepackage{placeins}
	\usepackage{changepage}
	%\hyphenation{con-fine-ment}
	
%%%% spezielle Formatierungen %%%%
	\setlength{\emergencystretch}{25pt} %verhindert das Herausragen von Wörtern übers Zeilenende
	\setlength{\parindent}{0pt} %Kein Einschub bei neuem Absatz
	\setlength{\parskip}{2pt plus 1pt} %Erhöht Abstand zwischen Absätzen (um 1pt flexibel bei Seitenumbrüchen)
	
	%%%% Dokument-Variablen %%%%
\date{\today}

%%%% Eigene Befehle %%%%
\DeclareGraphicsRule{*}{mps}{*}{}
	\newcommand{\mE}[1]{\,\mathrm{#1}} %Einheiten im Mathemodus
	\renewcommand{\sl}[1]{\slashed{#1}} %Feynman-Slash
	\newcommand{\dummyImage}[2]{  %Erzeugt eine Umgebung wie includegraphics
		\frame{\mbox{\rule{0pt}{#2}	Bild fehlt  noch \rule{#1}{0pt}}}
	}
	\renewcommand{\i}{\mathrm{i}} %Imaginäre Einheit
%	\renewcommand{\vec}[1]{\textbf{#1}} %Fette Vektoren
	\newcommand{\bc}{\begin{center}}
	\newcommand{\ec}{\end{center}}

\newcommand{\matM}{\mathcal{M}}
\newcommand{\matH}{\mathcal{H}}
\newcommand{\matS}{\mathcal{S}}
\newcommand{\matU}{\mathcal{U}}
\newcommand{\matUL}{\mathcal{U}_L}
\newcommand{\matUR}{\mathcal{U}_R}

%%% center all the figures and tables
\makeatletter
\g@addto@macro\@floatboxreset\centering
\makeatother
